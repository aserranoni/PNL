\PassOptionsToPackage{dvipsnames}{xcolor}

\documentclass[12pt,twoside,a4paper]{book}
\usepackage{packages}
\title{MAP5747 Programação Não Linear: Exercícios}
\author{Ariel Serranoni \\
        Chico Dias}
\date{2º semestre de 2019}

\begin{document}

\maketitle
\newpage
\tableofcontents
\newpage
\chapter*{Infromações da Disciplina}
\label{sec:intro}
\addcontentsline{toc}{section}{\nameref{sec:intro}}
\question{Informações Básicas}\label{new-question}

Estas são as notas de aula de Álgebra Linear(MAT5730), as aulas acontecem na sala B-134 às terças 10h e às quintas 8h.


\question{Informações do Professor}

O professor é o Ivan Shestakov, sua sala é a 290-A e o seu e-mail é shestak@ime.usp.br

\question{Bibliografia}
\nocite{*}
\bibliographystyle{plain}
\bibliography{samples}
\question{Avaliação}

A nota final da disciplina será a média aritimética de P1, P2, e P3. Todos os
alunos poderão fazer a prova sub para substituir a menor das suas notas (Sub
aberta). As datas das provas são as seguintes:

\begin{table}[h!]
  \begin{center}
    
    \label{tab:table1}
    \begin{tabular}{l|r} 
     \textbf{Prova} & \textbf{Data}\\
      \hline
      P1 & 10-09\\
      P2 & 15-10\\
      P3 & 12-11\\
      SUB & 19-11
    \end{tabular}
  \end{center}
\end{table}
\question{Outras Informações}
\begin{enumerate}[label=(\roman*)]
\item Teremos listas, que não contarão para a nota
\item As listas serão publicadas em 
\item Não haverá monitoria
\end{enumerate}
\newpage 

\chapter*{Lista 1}
\begin{problema}
Seja \(f\colon\Reals^n\to\Reals\) e sejam \(B\subseteq A\subseteq\Reals^n\). Se
\(\inf_{x\in\Reals^n}f(x)=\alpha\in\Reals\), então
\begin{enumerate}[label=(\roman*)]
\item \(\inf_{x\in A}f(x)\leq\inf_{x\in B}f(x)\);
\item todo minimizador de \(f\) em \(A\) é um minimizador de \(f\) em \(B\).
\end{enumerate}
\end{problema}
\begin{proof}[Solução]\hfill
  \begin{enumerate}[label=(\roman*)]
  \item
  \item
    \end{enumerate}
\end{proof}

\section{PNL Exercício 1 - Lista 1}
\label{sec:org41d184a}

a) \[\inf_{x\in A} f(x)=\min\{\inf_{x\in B}f(x), \inf_{x\in A\setminus B}f(x)\}\leq\inf_{x\in B}f(x).\]

b)

\section{PNL Exercício 2 - Lista 1}
\label{sec:org8affeec}

Considere a função \(f\colon\Reals\to\Reals\) dada por \(f(x)\coloneqq\exp(x)\). Considere \(\Omega=\Naturals\).
Então cada ponto \(\bar{x}\in\Omega\) minimiza \(f\) localmente e, como \(f\) é injetora temos que \(f(x)\not=f(y)\)
sempre que \(x\not= y\).

\section{PNL Exercício 3 - Lista 1}
\label{sec:orga567c7d}
Primeiramente, note que o conjunto \(f(\Omega)\) é compacto pois \(f\) é contínua e \(\Omega\) é compacto.

Vamos mostrar que \(\alpha\coloneqq\inf_{x\in\Omega}f(x)\in f(\Omega\)$\backslash$) e \(\beta\coloneqq\sup_{x\in\Omega}f(x)\in f(\Omega)\).
Como \(f(\Omega)\) é fechado temos que \(f(\Omega)=\overline{f(\Omega)}\). Portanto é suficiente mostrar que 
\(\alpha,\beta\in\overline{f(\Omega)}\). Seja \(\varepsilon\in\Reals_{++}\) e note que se \(\alpha +\varepsilon\mathbb{B}\cap f(\Omega)=\varnothing\) então \(\inf_{x\in\Omega}f(x)=\inf f(\Omega)\geq\alpha+\varepsilon\).
Isso implica que \(\inf f(\Omega) > \alpha\). Contradição. (escreve analogamente pra \(\beta\))


\section{PNL Exercício 4 - Lista 1}
\label{sec:org026ff9d}

Considere a função \(f\colon\Reals\setminus\{0\}\to\Reals\setminus\{0\}\) dada por \(f(x)\coloneqq\frac{1}{x}\). Se consideramos
\(\Omega=[-1,0)\), temos que \(f\) é contínua em \(\Omega\) e que \(\Omega\) é limitado, mas não fechado.
Portanto não vale o Teorema de Bolzano-Weierstrass e \(f\) não possui minimizador, de fato \(f\) é ilimitada em \(\Omega\).
Similarmente, se \(\Omega=[-1,0]\) temos que \(\Omega\) é compacto mas \(f\) não é contínua em \(\Omega\) e tb n vale o teorema
Ainda\ldots{}

\section{PNL Exercício 5 - Lista 1}
\label{sec:org5714b88}

Como \(f\) é contínua, temos que o conjunto de nível dado no enunciado é fechado. Além disso, temos
por hipótese que o conjunto é limitado. Assim, o resultado segue aplicando o exercicio 3

\section{PNL Exercício 6 - Lista 1}
\label{sec:orga03c240}

Seja \(x\in\Reals^n\) e considere o conjunto de nível \[N\coloneqq\{y\in\Reals^n\,\colon f(y)\leq f(x)\}.\]
Como \(f\) é contínua temos que \(N\) é fechado. Agora suponha que \(N\) não é limitado,
então existe uma sequencia \(\{y_n\}_{n\in\Naturals}\) tal que \(\|y_n\|\rightarrow\infty\) mas \(f(y_n)\leq f(x)\) para
todo \(n\in\Naturals\), o que contradiz a hipótese de que \(f\) é coerciva. Assim concluímos que \(N\) é compacto e o resultado 
segue do exercicio 3.

\section{PNL Exercício 7 - Lista 1}
\label{sec:org290910a}
a) Considere \(f(x)=\exp(x)\) e \(\Omega=\{0\}\)

b) Considere \(f(x)=-x^2\) e \(\Omega=\{0\}\)

c) Considere \(f(x)=x^3\) e \(\Omega=\Reals\)

d) Considere \(f(x)=x^3\) e \(\Omega=\Reals\)

\end{document}
