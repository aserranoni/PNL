\PassOptionsToPackage{dvipsnames}{xcolor}

\documentclass[12pt,twoside,a4paper]{article}
\usepackage{packages}
\title{MAP5747 Programação Não Linear: Exercícios}
\author{Ariel Serranoni}
\date{2º semestre de 2019}

\begin{document}

\maketitle

\section{Lista 1}

\begin{problema}
  Seja \(f\colon\Reals^n\to\Reals\) e sejam
  \(B\subseteq A\subseteq\Reals^n\). Se
 \(\inf_{x\in\Reals^n}f(x)=\alpha\in\Reals\), então
\begin{enumerate}[label=(\roman*)]
\item \(\inf_{x\in A}f(x)\leq\inf_{x\in B}f(x)\);
\item todo minimizador de \(f\) em \(A\) é um minimizador de \(f\)
  em \(B\).
\end{enumerate}
\end{problema}
\begin{proof}[Solução]\hfill
  \begin{enumerate}[label=(\roman*)]
  \item \[\inf_{x\in A} f(x)=
      \min\{\inf_{x\in B}f(x), \inf_{x\in A\setminus B}f(x)\}\leq
      \inf_{x\in B}f(x).\]
  \item Seja \(x\) tal que \(f(x)\leq f(y)\) para cada \(y\in A\). Como
    \(B\subseteq A\) temos que \(f(x)\leq f(y)\) para cada \(y\in B\).
    Logo,
    \(x\) minimiza \(f\) em \(B\).\qedhere
    \end{enumerate}
\end{proof}

\begin{problema}
Exercício 2 - Lista 1
\end{problema}
\begin{proof}[Solução]
  Considere a função \(f\colon\Reals\to\Reals\) dada por
  \(f(x)\coloneqq\exp(x)\). Considere \(\Omega=\Naturals\).
  Então cada ponto \(\bar{x}\in\Omega\) minimiza \(f\) localmente e,
  como \(f\) é injetora temos que \(f(x)\not=f(y)\)
  sempre que \(x\not= y\).\qedhere
\end{proof}

\begin{problema}
Exercício 3 - Lista 1 
\end{problema}
\begin{proof}[Solução]
 
Seja \(\{x_k\}_{k\in\Naturals}\subseteq \Omega\) uma sequência qualquer e
considere a sequência \(\{f(x_k)\}_{k\in\Naturals}\subseteq\Reals\). Como
\(\Omega\) é compacto temos que \(\{x_k\}_{k\in \Naturals}\) admite uma
subsequência convergindo para algum \(x\in\Omega\). Neste caso, segue que
\(\{f(x_k)\}_{k\in\Naturals}\) também admite uma subsequência convergindo
para \(f(x)\). Como \(x\in Omega\) temos que
\(f(x)\in f(\Omega)\). Mostramos
assim que cada sequencia em \(f(\Omega)\) admite uma
subsequência convergindo
para um elemento do próprio \(f(\Omega)\), ou seja, \(f(\Omega)\) é
compacto.

Finalmente, vamos mostrar que \(\alpha\coloneqq\inf_{x\in\Omega}f(x)\in
f(\Omega\)$\backslash$) e
\(\beta\coloneqq\sup_{x\in\Omega}f(x)\in f(\Omega)\).
Como \(f(\Omega)\) é fechado temos que
\(f(\Omega)=\overline{f(\Omega)}\). Portanto é suficiente mostrar que 
\(\alpha,\beta\in\overline{f(\Omega)}\). Seja
\(\varepsilon\in\Reals_{++}\) e note que se
\(\alpha +\varepsilon\mathbb{B}\cap f(\Omega)=\varnothing\)
então \(\inf_{x\in\Omega}f(x)=\inf f(\Omega)\geq\alpha+\varepsilon\).
Isso implica que \(\inf f(\Omega) > \alpha\).
Contradição. [escrevemos analogamente pra \(\beta\)].
\end{proof}

\begin{problema}
Exercício 4 - Lista 1
\end{problema}
\begin{proof}[Solução]
  Considere a função
  \(f\colon\Reals\setminus\{0\}\to\Reals\setminus\{0\}\)
  dada por \(f(x)\coloneqq\frac{1}{x}\). Se consideramos
  \(\Omega=[-1,0)\), temos que \(f\) é contínua em \(\Omega\) e
  que \(\Omega\) é limitado, mas não fechado.
  Portanto não vale o Teorema de Bolzano-Weierstrass e \(f\) não possui
  minimizador, de fato \(f\) é ilimitada em \(\Omega\).
  Similarmente, se \(\Omega=[-1,0]\) temos que \(\Omega\) é compacto
  mas \(f\) não é contínua em \(\Omega\) e tb n vale o teorema.
\end{proof}

\begin{problema}
Exercício 5 - Lista 1   
\end{problema}
\begin{proof}[Solução]
  Como \(f\) é contínua, temos que o conjunto de nível
  dado no enunciado é fechado. Além disso, temos
  por hipótese que o conjunto é limitado. Assim, o
  resultado segue aplicando o exercicio 3.
\end{proof}

\begin{problema}
Exercício 6 - Lista 1  
\end{problema}
\begin{proof}[Solução]
  Seja \(x\in\Reals^n\) e considere o conjunto de nível
  \[N\coloneqq\{y\in\Reals^n\,\colon f(y)\leq f(x)\}.\]
Como \(f\) é contínua temos que \(N\) é fechado. Agora suponha que
\(N\) não é limitado,então existe uma sequencia \(\{y_n\}_{n\in\Naturals}\)
tal que \(\|y_n\|\rightarrow\infty\) mas \(f(y_n)\leq f(x)\) para
todo \(n\in\Naturals\), o que contradiz a hipótese de que
\(f\) é coerciva. Assim concluímos que \(N\) é compacto e
o resultado segue do exercicio 3.
\end{proof}

\begin{problema}
Exercício 7 - Lista 1
\end{problema}
\begin{proof}[Solução]\hfill
  \begin{enumerate}
\item Considere \(f(x)=\exp(x)\) e \(\Omega=\{0\}\).
\item Considere \(f(x)=-x^2\) e \(\Omega=\{0\}\).
\item Considere \(f(x)=x^3\) e \(\Omega=\Reals\).
\item Considere \(f(x)=x^3\) e \(\Omega=\Reals\).
\end{enumerate}
\end{proof}
\section{Lista 1 - Old}
\begin{problema}\label{rosenmin}
Exercício 2
\end{problema}
\begin{proof}[Solução]
  Iniciamos calculando uma forma polinomial para a função \(f\).
  Daí, obtemos que
  \begin{equation}\label{rosenfunc}
   f(x_1,x_2)=100x_1^4+x_1^2-2x_1+100x_2^2-200x_1^2x_2+1. 
  \end{equation}
   Além disso, vamos calcular o vetor gradiente e a matriz hessiana de \(f\):
\begin{equation}\label{gradrosen}
  \nabla f(x_1,x_2)=\begin{pmatrix}
    400x_1^3 + 2x_1-400x_1x_2 -2 \\
    200 x_2 - 200x_1^2
  \end{pmatrix}
  \text{ e } \nabla^2 f(x_1,x_2)=\begin{pmatrix}
    1200x_1^2+2-400x_2 & -400x_1 \\
    -400x_1 & 200
    \end{pmatrix}.
\end{equation}
Resolvendo o sistema \(\nabla f(x_1,x_2)=0\) nos dá a solução única
\(x\coloneqq(1,1)^\top\). Feito isso verificamos que
\[\nabla^2f(1,1)=\begin{pmatrix}
    802 & -400 \\
    -400 & 200 \end{pmatrix}\in\mathbb{S}^n_{++}.\]
Assim, concluímos que \(x\) é o único minimizador global de \(f\).
\end{proof}


\section{Lista 2 - Old}
\begin{problema}
Exercício  
\end{problema}
\begin{proof}[Solução]
  Neste exercício faremos uso das contas feitas no
  Exercício \ref{rosenmin}.
  \begin{enumerate}
  \item Primeiro, veja que \(d=-\nabla f(0,0)=(2,0)^\top\). Neste caso
    segue que
    \begin{align}
      \phi(\alpha)=f(0+\alpha d)&=f((2\alpha,0)^\top)\\&=
      100((-2\alpha)^2)^2+(1-2\alpha^2)\\&=
      100(16\alpha^4)+4\alpha^2-4\alpha + 1.
    \end{align}
  \item Primeiro, vamos calcular a direção de Newton. Por definição, segue
    \
  \end{enumerate}
\end{proof}

\end{document}
