\PassOptionsToPackage{dvipsnames}{xcolor}

\documentclass[12pt,twoside,a4paper]{book}
\usepackage{packages}
\title{MAP5747 Programação Não Linear: Exercícios}
\author{Ariel Serranoni \\
        Chico Dias}
\date{2º semestre de 2019}

\begin{document}

\maketitle
\newpage
\tableofcontents
\newpage
\chapter*{Infromações da Disciplina}
\label{sec:intro}
\addcontentsline{toc}{section}{\nameref{sec:intro}}
\question{Informações Básicas}\label{new-question}

Estas são as notas de aula de Álgebra Linear(MAT5730), as aulas acontecem na sala B-134 às terças 10h e às quintas 8h.


\question{Informações do Professor}

O professor é o Ivan Shestakov, sua sala é a 290-A e o seu e-mail é shestak@ime.usp.br

\question{Bibliografia}
\nocite{*}
\bibliographystyle{plain}
\bibliography{samples}
\question{Avaliação}

A nota final da disciplina será a média aritimética de P1, P2, e P3. Todos os
alunos poderão fazer a prova sub para substituir a menor das suas notas (Sub
aberta). As datas das provas são as seguintes:

\begin{table}[h!]
  \begin{center}
    
    \label{tab:table1}
    \begin{tabular}{l|r} 
     \textbf{Prova} & \textbf{Data}\\
      \hline
      P1 & 10-09\\
      P2 & 15-10\\
      P3 & 12-11\\
      SUB & 19-11
    \end{tabular}
  \end{center}
\end{table}
\question{Outras Informações}
\begin{enumerate}[label=(\roman*)]
\item Teremos listas, que não contarão para a nota
\item As listas serão publicadas em 
\item Não haverá monitoria
\end{enumerate}
\newpage 

\chapter*{Lista 1}
\begin{problema}
Seja \(f\colon\Reals^n\to\Reals\) e sejam \(B\subseteq A\subseteq\Reals^n\). Se
\(\inf_{x\in\Reals^n}f(x)=\alpha\in\Reals\), então
\begin{enumerate}[label=(\roman*)]
\item \(\inf_{x\in A}f(x)\leq\inf_{x\in B}f(x)\);
\item todo minimizador de \(f\) em \(A\) é um minimizador de \(f\) em \(B\).
\end{enumerate}
\end{problema}
\begin{proof}[Solução]\hfill
  \begin{enumerate}[label=(\roman*)]
  \item
  \item
    \end{enumerate}
\end{proof}
\end{document}